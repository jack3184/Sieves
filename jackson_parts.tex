%% problem_set.tex

\documentclass[10pt]{article}

\usepackage{fullpage} % normal margins
\usepackage{amsmath} % for align
\usepackage{graphicx}

\begin{document}
\section*{Estimation}

\paragraph{CV optimal K} \mbox{} \\

\textbf{Insert comments like:
\begin{itemize}
\item CV is chosen by...
\item We find optimal CV to be: (see next section for relationship to K)
\end{itemize}}

With optimal K, our model matches the data well: (The red line is the known expected labour supply function. The green
line is observed hours worked, and the blue line is our fitted model.)


\includegraphics[scale=.8]{plot1.pdf}





\paragraph{Policy simulation} \mbox{} \\
Following sections 3.4 and 5, we estimate the percent change in
average labor supply from shifting budget sets. We assume a minimum
wage equal to $\tfrac{4}{10}$ of the mean wage is introduced:
\begin{equation*}
  w_a = \max\left\{\tfrac{4}{10} \mathrm{E}[w_b], w_b\right\}
\end{equation*}
Where, as in B\&N, subscripts $a$ and $b$ refer to after and before
the policy change. We replicate Table I from section 5.3:

% latex table generated in R 3.3.2 by xtable 1.8-2 package
% Wed Mar 01 21:50:31 2017
\begin{table}[ht]
\centering
\begin{tabular}{lllll}
  \hline
K & Additional.term & CV & M.hat & M.SE \\ 
  \hline
3 & 0.1 & 0.6675 & 0.0015 & 0 \\ 
  4 & 1.0.0 & 0.6895 & 0.0013 & 1e-04 \\ 
  5 & 0.1.0 & 0.6939 & 0.0064 & 9e-04 \\ 
  6 & 2.0 & 0.693 & 0.0064 & 9e-04 \\ 
  7 & 1.1 & 0.6938 & 0.0067 & 9e-04 \\ 
  8 & 0.2 & 0.7985 & 0.0058 & 5e-04 \\ 
  9 & 2.0.0 & 0.7977 & 0.0058 & 5e-04 \\ 
  10 & 1.1.0 & 0.7975 & 0.0106 & 3e-04 \\ 
  11 & 0.2.0 & 0.7969 & 0.0116 & 0.001 \\ 
  12 & 1.0.1 & 0.8841 & -0.0554 & 9e-04 \\ 
  13 & 3.0 & 0.8837 & -0.0554 & 8e-04 \\ 
  14 & 2.1 & 0.8833 & -0.055 & 8e-04 \\ 
  15 & 1.2 & 0.8851 & -0.059 & 6e-04 \\ 
  16 & 0.3 & 0.8896 & -0.0815 & 2e-04 \\ 
  17 & 3.0.0 & 0.8886 & -0.0816 & 1e-04 \\ 
  18 & 2.1.0 & 0.8891 & 0.2236 & 2e-04 \\ 
  19 & 1.2.0 & 0.8886 & 0.2433 & 2e-04 \\ 
  20 & 0.3.0 & 0.8883 & 0.2938 & 3e-04 \\ 
  21 & 2.0.1 & 0.889 & 0.2983 & 3e-04 \\ 
   \hline
\end{tabular}
\end{table}
 \mbox{} \\
As the first row of B\&N's table, the regression in our first row has
regressors $(1,y_j,w_j)$. Working down the rows, additional terms are
added to the basis. Entries with two digits indicate terms
$y_j^{\text{digit }1}w_j^{\text{digit }2}$ in the power series, and
three digits indicate $(y_{j}^{\text{digit }1}w_j^{\text{digit }2} -
y_{j+1}^{\text{digit }1}w_{j+1}^{\text{digit }2})l_j^{\text{digit }3}$
as in equations (3.2)-(3.4). 

The estimator M can be computed in a simpler way, but we have used
the formulas provided by B\&N. SE are computed similarly. Compared to
B\&N, our results are less robust to the choice of basis function,
but, plainly, there are so many differences between our analyses, that
comparisions are not very clarifying.

We decided not do include the elasticity functional $E$, since it
would be very tedious to take derivatives of the basis function.

\paragraph{A comment on constructing the basis}
We encountered three issues constructing a sieve space using the power
series form (section 3.1). The basic idea was to increase the degree
of the power series approximation (recalling the approximation is a
sum of two power series), by adding individual terms as K increases.

The first issue is there are multiple terms of the same degree, so
there is no natural way to add terms. For example, $x^2y^3$ and
$x^4y^1$ are both degree 5, so when it comes to adding an extra degree
5 term, which should be added? We just picked arbitrarily, as can be
seen in the above table.

The second issue regards linear dependence. Quite often additional
columns of the basis matrix were linear dependent with existing
columns. Our solution to this was to exclude the `higher' order linearly
dependent columns. (Strictly speaking, we constructed the basis by
adding columns only if the additional column did not belong to the
existing column space.)

The last issue is about numerical constraints. I found that, due to
the large powers being taken, the determinant of the square matrix
generated by the basis matrix was very large. I believe this made the
matrix inversion numerically imprecise. However this is only an issue
when the number of terms becomes large ($approx$ 25).

\end{document}
